\documentclass[11pt]{article}
\usepackage[margin=1in]{geometry} 
\geometry{letterpaper}   

\usepackage{amsmath}
\usepackage{amssymb,amsfonts,bbm,mathrsfs,stmaryrd}
\usepackage{url}

%%% Theorems and references %%%
\usepackage[amsmath,thmmarks]{ntheorem}
\usepackage{hyperref}
\usepackage{cleveref}

\theoremstyle{change}

\newtheorem{defn}[equation]{Definition}
\newtheorem{definition}[equation]{Definition}
\newtheorem{thm}[equation]{Theorem}
\newtheorem{theorem}[equation]{Theorem}
\newtheorem{prop}[equation]{Proposition}
\newtheorem{proposition}[equation]{Proposition}
\newtheorem{lemma}[equation]{Lemma}
\newtheorem{cor}[equation]{Corollary}
\newtheorem{conj}[equation]{Conjecture}
\newtheorem{conjecture}[equation]{Conjecture}
\newtheorem{exercise}[equation]{Exercise}
\newtheorem{example}[equation]{Example}

\theorembodyfont{\upshape}
\theoremsymbol{\ensuremath{\Diamond}}
\newtheorem{eg}[equation]{Example}
\newtheorem{remark}[equation]{Remark}

\theoremstyle{nonumberplain}

\theoremsymbol{\ensuremath{\Box}}
\newtheorem{proof}{Proof}

\qedsymbol{\ensuremath{\Box}}

\creflabelformat{equation}{#2(#1)#3} 

\crefname{equation}{equation}{equations}
\crefname{eg}{example}{examples}
\crefname{defn}{definition}{definitions}
\crefname{prop}{proposition}{propositions}
\crefname{thm}{Theorem}{Theorems}
\crefname{lemma}{lemma}{lemmas}
\crefname{cor}{corollary}{corollaries}
\crefname{remark}{remark}{remarks}
\crefname{section}{Section}{Sections}
\crefname{subsection}{Section}{Sections}

\crefformat{equation}{#2equation~(#1)#3} 
\crefformat{eg}{#2example~#1#3} 
\crefformat{defn}{#2definition~#1#3} 
\crefformat{prop}{#2proposition~#1#3} 
\crefformat{thm}{#2Theorem~#1#3} 
\crefformat{lemma}{#2lemma~#1#3} 
\crefformat{cor}{#2corollary~#1#3} 
\crefformat{remark}{#2remark~#1#3} 
\crefformat{section}{#2Section~#1#3} 
\crefformat{subsection}{#2Section~#1#3} 

\Crefformat{equation}{#2Equation~(#1)#3} 
\Crefformat{eg}{#2Example~#1#3} 
\Crefformat{defn}{#2Definition~#1#3} 
\Crefformat{prop}{#2Proposition~#1#3} 
\Crefformat{thm}{#2Theorem~#1#3} 
\Crefformat{lemma}{#2Lemma~#1#3} 
\Crefformat{cor}{#2Corollary~#1#3} 
\Crefformat{remark}{#2Remark~#1#3} 
\Crefformat{section}{#2Section~#1#3} 
\Crefformat{subsection}{#2Section~#1#3} 


\numberwithin{equation}{section}


%%% Tikz stuff %%%

\usepackage{tikz}
\tikzset{dot/.style={circle,draw,fill,inner sep=1pt}}
\usepackage{braids}
\usepackage{tqft}
\usetikzlibrary{tqft}
\usetikzlibrary{cd}
\usetikzlibrary{arrows}
%%% Letters, Symbols, Words %%%

\newcommand\Aa{{\cal A}}
\newcommand\Oo{{\cal O}}
\newcommand\Uu{{\cal U}}
\newcommand\NN{{\mathbb N}}
\newcommand\RR{{\mathbb R}}
\newcommand\Ddd{\mathscr{D}}
\renewcommand{\d}{{\,\rm d}}
\newcommand\T{{\rm T}}

\newcommand\mono{\hookrightarrow}
\newcommand\sminus{\smallsetminus}
\newcommand\st{{\textrm{ s.t.\ }}}
\newcommand\ket[1]{\mid #1 \rangle}
\newcommand\bra[1]{\langle #1 \mid}
\newcommand\setof[1]{\{ #1 \}}
\newcommand\lt{<}
\newcommand\abs[1]{ \mid #1 \mid }
\newcommand\conjbar[1]{\overline{#1}}

\DeclareMathOperator{\Aut}{Aut}
\DeclareMathOperator{\dVol}{dVol}
\DeclareMathOperator{\ev}{ev}
\DeclareMathOperator{\fiber}{fiber}
\DeclareMathOperator{\GL}{GL}
\DeclareMathOperator{\id}{id}
\DeclareMathOperator{\sign}{sign}
\DeclareMathOperator{\tr}{tr}


\title{README}
\author{Ammar Husain}

\begin{document}
\maketitle

\section{Modular Forms}

\begin{definition}[Modular/Cusp Forms]
$S_k (N)$ transform with a factor of $(c \tau + d)^k$ under $\Gamma_0 (N)$ where $c \equiv 0$ modulo $N$. They must vanish at $\infty$. $M_k (N)$ don't have to vanish at $\infty$. In terms of $q = e^{2 \pi i \tau}$ $S_k$ gives a q expansion holomorphic on the disc vanishing at the origin. $M_k (N)$ doesn't have to vanish at the origin.
\end{definition}

Acting on $M_k (N=1)$. That is $\Gamma_0 (1) = SL(2,\mathbb{Z})$ without $N$ condition on $c$ entry and $k$ gives the weight.

For $r \geq 1$ and $(n,m)=1$

\begin{eqnarray*}
T_{nm} &=& T_n T_m = T_m T_n\\
T_{p^{r+1}} &=& T_{p^r} T_p - p^{k-1} T_{p^{r-1}}\\
T_{p^2} &=& T_{p} T_p - p^{k-1}\\
T_{p^3} &=& (T_{p} T_p - p^{k-1} ) T_p - p^{k-1} T_p\\
&=& T_p^3 - p^{k-1} T_p - p^{k-1} T_p\\
T_{p^4} &=& (T_p^3 - 2 p^{k-1} T_p) T_p - p^{k-1} (T_{p}^2 - p^{k-1})\\
&=& T_p^4 - 3 p^{k-1} T_p^2 + (p^{k-1})^2\\
\end{eqnarray*}

\begin{cor}
For a general $T_n$ one can factor $n$ and then use these identities to reduce to a polynomial in the $T_p$ for only primes $p$.
\end{cor}

\begin{proof}
\begin{eqnarray*}
n &=& p_1^{a_1} p_2^{a_2} \cdots\\
T_n &=& T_{p_1^{a_1}} \cdots\\
&=& (T_{p_1^{a_1-1}} T_p - p^{k-1} T_{p_1^{a_1-2}}) \cdots\\
\end{eqnarray*}
\end{proof}

\begin{eqnarray*}
f &\equiv& \sum_{m=0}^\infty c_m q^m\\
T_p f &=& \sum_{\mu=0}^\infty c_{p\mu} q^{\mu} + \sum_{\nu=0}^\infty c_\nu q^{p \nu}
\end{eqnarray*}

If $f \in S_k (1)$, then rescale so starts with $1*q$.

Now looking for Hecke eigenforms. If we know $T_p$ is acting by a scalar $\lambda_p$, then we know that $T_{p^r}$ are acting by scalars as well

\begin{eqnarray*}
T_{p^2} f &=& ( \lambda_p^2 - p^{k-1} )f\\
\end{eqnarray*}

\begin{theorem}[Eichler-Selberg]
Formula for trace of $T_n$ on $S_k (1)$.
\end{theorem}

\begin{cor}
If we know all the traces for $T_{p^n}$ up to $d-1$ where $d$ is dimension of the $S_k (1)$ we can find characteristic polynomial for $T_p$.
\end{cor}

\begin{proof}
The first term of $T_{p^r}$ is $T_p^r$ so knowledge of traces of $T_{p^r}$ can be backsubstituted to recover traces of $T_p^r$ then use formula for characteristic polynomial in terms of traces of powers.
This is what CharPolyHelper does in the Mathematica file. It takes the list of $\tr T_{p^r}$ and outputs the list of $\tr T_p^r$. CharPolyHelper2 is supposed to transfer that information back into the characteristic polynomial using formula of $\tr A^r$ in terms of symmetric functions of eigenvalues back to the characteristic polynomial which has those eigenvalues as roots.
\end{proof}

\section{Point Counts}

\begin{definition}[Zeta Function]
\begin{eqnarray*}
Z (C , u ) &=& e^{\sum_{m=0}^\infty \frac{N_m}{m} u^m}\\
\end{eqnarray*}

where $N_m$ is the point count over $q^m$.
\end{definition}

\begin{thm}[Weil]
\begin{eqnarray*}
Z (C,u) &=& \frac{P(u)}{(1-u)(1-q u)}
\end{eqnarray*}

where $P$ is a polynomial that can be written as

\begin{eqnarray*}
P(u) &=& \prod_{i=1}^{2g} (1- \omega_i u)\\
&=& 1 + \sum e_k (-u)^k
\end{eqnarray*}

\end{thm}

\begin{cor}
If you are given $N_m$ for $m=1 \cdots 2g$, then can recover the $e_k$ symmetric functions of the $\omega_i$. This is SolveForEks in the Mathematica notebook.

Once you have that, you can plug that back in and recover the count over $q^m$ for even higher $m$ without solving that much harder equation. This is GiveMthPointCount.

If the $\omega_i$ are provided, the point counts over all $q^m$ for $1 \cdots m_{max}$ are given through SolveForNms.
\end{cor}

\begin{theorem}[Hasse-Weil Bound]
\begin{eqnarray*}
\abs{N_{m} - (q^m + 1)} &\leq& 2 g \sqrt{q^m}
\end{eqnarray*}

So we may write $N_m = q^m + 1 + \epsilon_m$ where $\abs{\epsilon_m} \leq 2 g \sqrt{q^m}$. This way instead of providing the $N_m$ data, one can provide the smaller $\epsilon_m$ data instead.

\end{theorem}

\begin{definition}[$Z_{mot,Kapranov}$]
\begin{eqnarray*}
Z_{mot,Kapranov} (X,u) &=& \sum u^n [Sym^n X]\\
\end{eqnarray*}

where $[Sym^n X]$ is the motive of $Sym^n X$. In particular, can give the points from the motive. Call that map $\mu$.

\begin{eqnarray*}
\mu Z_{mot,Kapranov} (X,u) &=& Z(X,u)
\end{eqnarray*}

\end{definition}

\begin{cor}
This means we can recover the point counts of $Sym^n C$ from the point counts $N_m$ that were given. This is done in symmetricPowerCounts1. In particular, you can get the point count for $Sym^g C$. That is given by the function jacobianSize1.
\end{cor}

\section{Crypto}

\begin{definition}[Discrete Logarithm]
Let $C$ be a finite cyclic group with generator $g_0$. $f \; C \to \mathbb{N}$ an injective set map.

For $a \in f(C) \subset \mathbb{N}$, solve for

\begin{eqnarray*}
\log_{g_0} (a) &=& \text{min} \; \;  \setof { n \mid n \leftarrow \mathbb{N} , f (g_0^n) = a }\\
\end{eqnarray*}

\end{definition}

\begin{definition}[Abelian Variety]
A connected projective algebraic group over the base field $k$. The group law on the points becomes abelian group.
\end{definition}

\begin{definition}[Dual abelian variety]
For $A$, there is a dual $A^\vee$. This has the universal property for parameterizing families of degree $0$ line bundles.
\end{definition}

\begin{cor}[Weil Pairing]
Take the $n$ torsion in both $A$ and $A^\vee$. They are Cartier duals. If $A$ is an elliptic curve, this is Weil pairing.
\end{cor}

\begin{remark}
Embedding arbitrary abelian variety into a Jacobian. After allowing yourself an infinite field like $\bar{\mathbb{F}_p}$.

\url{https://mathoverflow.net/questions/304314/is-every-abelian-variety-a-subvariety-of-a-jacobian}

There is the map $M_g \to A_g$ from the moduli space of genus $g$ curves to the moduli space of abelian varieties of dimension $g$. It takes curves of genus $g$ to their Jacobians. The Schottky problem asks to characterize this image.
\end{remark}

So pick a point in this abelian variety and see the cyclic group it generates. The fact that the structure maps were all algebraic makes the operations you need to do over and over again efficient.

In particular if the curve is an elliptic curve, really should think of that group law as happening on Jacobian of an elliptic curve, but with the coincidence that that is elliptic curve again. However, this tells you how it generates for higher genus.

\begin{definition}[CM Type]
An abelian variety is of CM type when $End (A) \otimes \mathbb{Q}$ contains a commutative subring of dimension $2d$ over $\mathbb{Q}$. For $d=1$, this means that $End(A)$ is an order in an imaginary quadratic field.
\end{definition}

\subsection{Elliptic Curves}

\begin{definition}[Weirstrauss form]
\begin{eqnarray*}
y^2 &=& x^3 + a x + b
\end{eqnarray*}

Points are stored as $(x,y,isInfinity)$ triples. If $isInfinity=True$ the $x,y$ don't matter.

\end{definition}

\begin{definition}[Legendre form]
\begin{eqnarray*}
y^2 &=& x (x-1) (x - j)
\end{eqnarray*}

Points are stored as $(x,y,isInfinity)$ triples. If $isInfinity=True$ the $x,y$ don't matter.
\end{definition}

\begin{lemma}[Conversion]
Suppose we have a solution in Weirstrauss form with certain $a,b$.

\begin{eqnarray*}
x^3 + a x + b &=& (x - x_1) (x - x_2) (x - x_3)\\
\tilde{x} &=& x - x_1\\
x^3 + a x + b &\to& \tilde{x} ( \tilde{x} + x_1 - x_2) (\tilde{x} + x_1 - x_3)\\
X &=& \frac{1}{(x_2 - x_1)}  \tilde{x}\\
x^3 + a x + b &\to& (x_2 - x_1)^{3} X ( X - 1) (X + \frac{x_1 - x_3}{x_2 - x_1})\\
Y &=& (x_2 - x_1)^{3/2} y\\
Y^2 &=& X ( X - 1) (X + \frac{x_1 - x_3}{x_2 - x_1})\\
\end{eqnarray*}

So have the corresponding elliptic curve in Legendre form and the new expression for the same point as $X,Y$.

\end{lemma}

\url{http://www.hyperelliptic.org/EFD/g1p/auto-shortw-jacobian.html}

\begin{definition}[Jacobi embedding]
\end{definition}

\begin{lemma}[Addition in Weirstrauss]
Given via the function .. in the ipynb
\end{lemma}

\begin{lemma}[Addition in Legendre]
Given via the function .. in the ipynb
\end{lemma}

\section{Sophie Germain}

For the SophieGermain.ipynb

\section{Continued Fractions}

Given an $x$ and $maxDepth$, RationalFracToModular in ContinuedFraction.nb turns $x$ into a continued fraction and then gives the associated element in $Y \in SL(2,\mathbb{Z})$. $Y$ acting on the point at $\infty$ with the usual fractional linear transformation action returns $x$.

The graph is of one of the entries of $Y$ as a function of $x \in (0,1)$.

\end{document}